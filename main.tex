\documentclass[a4paper,12pt]{article}
\usepackage[utf8]{inputenc}
\usepackage{amsmath}
\usepackage{xcolor}

\definecolor{backcolour}{rgb}{0.95,0.95,0.92}
\definecolor{codegreen}{rgb}{0,0.6,0}
\definecolor{magenta}{rgb}{1,0,1}
\definecolor{codegray}{rgb}{0.5,0.5,0.5}
\definecolor{codepurple}{rgb}{0.58,0,0.82}
\usepackage{amssymb}
\usepackage{listings}
\usepackage{graphicx}
\usepackage{hyperref}
\usepackage{listings}
\usepackage{xcolor}
\usepackage{amsmath}

\lstset{
    basicstyle=\ttfamily\footnotesize,
    backgroundcolor=\color{backcolour},
    commentstyle=\color{codegreen},
    keywordstyle=\color{magenta},
    numberstyle=\tiny\color{codegray},
    inputencoding=utf8,
    extendedchars=true,
    breaklines=true,
    captionpos=b,
    keepspaces=true,
    numbers=left,
    numbersep=5pt,
    showspaces=false,
    showstringspaces=false,
    showtabs=false,
    tabsize=2,
    morecomment=[l]{//},
    morecomment=[s]{/*}{*/},
    morecomment=[s]{"""}{"""}
}

\title{Documentação sobre Protocolos de Segurança: SSL, TLS e HTTPS}
\author{Seu Nome}
\date{\today}

\begin{document}

\maketitle

\section{Introdução}
Os protocolos SSL (Secure Sockets Layer), TLS (Transport Layer Security) e HTTPS (HyperText Transfer Protocol Secure) são fundamentais para a segurança na comunicação digital. Eles garantem confidencialidade, integridade e autenticação dos dados transmitidos na internet, protegendo contra interceptações e ataques cibernéticos.

\section{Versões dos Protocolos e Melhorias}

\subsection{SSL - Secure Sockets Layer}
O SSL foi desenvolvido inicialmente pela Netscape nos anos 90 para garantir segurança nas comunicações pela internet. Ele passou por três versões principais antes de ser substituído pelo TLS:

\begin{itemize}
\item \textbf{SSL 1.0} - Nunca foi lançado publicamente devido a falhas de segurança graves.
\item \textbf{SSL 2.0} - Lançado em 1995, apresentava melhorias, mas ainda vulnerável a vários ataques.
\item \textbf{SSL 3.0} - Introduzido em 1996, corrigiu diversas falhas do SSL 2.0, mas ainda possuía vulnerabilidades como o ataque POODLE.
\end{itemize}

Devido às falhas de segurança do SSL, ele foi oficialmente descontinuado e substituído pelo TLS.

\subsection{TLS - Transport Layer Security}
O TLS foi desenvolvido como uma versão mais segura do SSL e passou por diversas atualizações:

\begin{itemize}
\item \textbf{TLS 1.0 (1999)} - Baseado no SSL 3.0, mas com melhorias na segurança e resistência a ataques.
\item \textbf{TLS 1.1 (2006)} - Introduziu proteções contra ataques de criptografia, como CBC (Cipher Block Chaining).
\item \textbf{TLS 1.2 (2008)} - Adicionou suporte para novos algoritmos de criptografia, como AES e SHA-2, tornando a comunicação mais segura.
\item \textbf{TLS 1.3 (2018)} - Melhorou a eficiência da comunicação, reduzindo a complexidade do handshake e removendo algoritmos inseguros como RC4 e SHA-1.
\end{itemize}

O TLS 1.3 é atualmente a versão mais segura e recomendada para garantir uma comunicação protegida na internet.

\subsection{HTTPS - HyperText Transfer Protocol Secure}
O HTTPS é a implementação do HTTP sobre uma camada segura de TLS, garantindo que os dados transmitidos sejam protegidos contra ataques. Ele segue a evolução do TLS e atualmente suporta as versões mais seguras desse protocolo.

Os principais benefícios do HTTPS incluem:
\begin{itemize}
\item \textbf{Confidencialidade:} Os dados são criptografados, impedindo que terceiros os leiam.
\item \textbf{Autenticação:} O uso de certificados digitais garante que o usuário está se comunicando com um servidor legítimo.
\item \textbf{Integridade:} A tecnologia impede que os dados sejam alterados durante a transmissão.
\end{itemize}

O funcionamento do HTTPS ocorre da seguinte maneira:
\begin{enumerate}
\item O usuário acessa um site que utiliza HTTPS.
\item O servidor envia seu certificado digital, garantindo sua autenticidade.
\item O navegador verifica o certificado e estabelece uma conexão segura utilizando TLS.
\item Todos os dados transmitidos entre o usuário e o servidor são criptografados, evitando que sejam interceptados.
\end{enumerate}

\section{Etapas de Segurança e Algoritmos}

\subsection{Autenticação e Troca de Chaves}
O processo de estabelecimento de uma conexão segura segue as seguintes etapas:

\begin{enumerate}
\item \textbf{Handshake SSL/TLS:}
\begin{itemize}
\item Cliente e servidor negociam versões e algoritmos suportados.
\item O servidor envia seu certificado digital (emitido por uma Autoridade Certificadora).
\item O cliente verifica a validade do certificado e, caso válido, procede com a troca de chaves.
\end{itemize}
\item \textbf{Troca de Chaves:}
\begin{itemize}
\item Algoritmos como RSA (Rivest-Shamir-Adleman) ou Diffie-Hellman são usados para gerar e trocar chaves seguras.
\end{itemize}
\item \textbf{Criptografia dos Dados:}
\begin{itemize}
\item Os dados são protegidos com algoritmos simétricos como AES (Advanced Encryption Standard).
\end{itemize}
\item \textbf{Integridade dos Dados:}
\begin{itemize}
\item Uso de funções hash (SHA-2, SHA-3) e códigos de autenticação de mensagem (HMAC) para verificar a integridade da comunicação.
\end{itemize}
\end{enumerate}

\section{Explicação do Código de Implementação em Python}

O código a seguir demonstra a implementação de um servidor HTTPS e um cliente HTTPS em Python, utilizando a biblioteca \textit{http.server} e \textit{requests}. O servidor é configurado para rodar localmente na porta 4443 e o cliente realiza uma requisição HTTPS para o servidor, verificando a autenticidade do certificado digital.
\begin{lstlisting}[language=Python]
    import http.server
    import ssl
    import requests
    import socket
    from requests.exceptions import SSLError
    import threading
    import time
\end{lstlisting}

O código a seguir define um manipulador de requisições HTTPS que envia uma resposta simples ao cliente. Ele é responsável por tratar as requisições GET recebidas pelo servidor. A resposta enviada ao cliente é "Conexao segura estabelecida via HTTPS!". 
\begin{lstlisting}[language=Python]
    class HTTPSRequestHandler(http.server.SimpleHTTPRequestHandler):
        def do_GET(self):
            self.send_response(200)
            self.send_header('Content-type', 'text/plain')
            self.end_headers()
            self.wfile.write(b'Conexao segura estabelecida via HTTPS!')
\end{lstlisting}

O comando a seguir define a função \textit{run\_server()}, que configura o servidor HTTPS na porta 4443. O servidor é configurado para utilizar um certificado digital autoassinado (\textit{server.pem}) e uma chave privada (\textit{server.key}). O contexto SSL é definido com o certificado e a chave, e o modo de segurança é ativado para o servidor. Por fim, o servidor é iniciado e fica aguardando requisições.
\begin{lstlisting}[language=Python]
    def run_server():
        # Configurando o servidor HTTP
        server_address = ('localhost', 4443)
        httpd = http.server.HTTPServer(server_address, HTTPSRequestHandler)
        
        # Definindo o contexto SSL
        context = ssl.SSLContext(ssl.PROTOCOL_TLS_SERVER)
        context.load_cert_chain(certfile='server.pem', keyfile='server.key')
    
        # Ativando o modo de seguranca
        httpd.socket = context.wrap_socket(httpd.socket, server_side=True)
        print("Servidor HTTPS rodando em https://localhost:4443")
        httpd.serve_forever()
\end{lstlisting}

O código a seguir define a função \textit{run\_client()}, que realiza uma requisição HTTPS para o servidor configurado. A URL do servidor é definida como "https://localhost:4443". A função \textit{requests.get()} é utilizada para realizar a requisição, e o certificado do servidor (\textit{server.pem}) é verificado. Caso o certificado seja inválido, uma exceção \textit{SSLError} é lançada e a conexão é rejeitada.
\begin{lstlisting}[language=Python]
    def run_client():
        # Configurando o cliente HTTPS
        url = "https://localhost:4443"
        try:
            #response = requests.get(url, verify=False)  # Desativa a verificacao SSL (somente para testes!)
            response = requests.get(url, verify="server.pem")
            print("Resposta do servidor:", response.text)
        # Tratamento de excecao para certificado invalido
        except SSLError:
            raise ValueError("Certificado invalido! Conexao rejeitada.")
            sys.exit(1)
\end{lstlisting}

O código a seguir inicia o servidor HTTPS em uma \textit{thread} separada e aguarda 2 segundos para garantir que o servidor esteja rodando antes de iniciar o cliente. O servidor é iniciado chamando a função \textit{run\_server()} e o cliente é iniciado chamando a função \textit{run\_client()}.
\begin{lstlisting}[language=Python]
    # Criando um certificado autoassinado
    server_thread = threading.Thread(target=run_server)
    server_thread.daemon = True
    server_thread.start()
    time.sleep(2)  # Espera o servidor iniciar
    run_client()
\end{lstlisting}

\section{Conclusão}
A adoção de protocolos como TLS e HTTPS é essencial para garantir segurança na internet. A evolução dessas tecnologias reforça a proteção contra ataques e vulnerabilidades, tornando a comunicação digital mais confiável e eficiente. O uso de certificados digitais e algoritmos de criptografia avançados assegura que as informações trocadas entre cliente e servidor permaneçam seguras e invioláveis.

\end{document}